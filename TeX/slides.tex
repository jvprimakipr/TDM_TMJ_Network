\documentclass{beamer}

%------------------------------------
%------------Libraries---------------
%------------------------------------

\usepackage[brazil]{babel}
\usepackage[utf8]{inputenc}
\usepackage{xpatch}
\usepackage{ragged2e}
\usepackage{xcolor}
\usepackage{url, hyperref}
\usepackage[portuguese,ruled,vlined]{algorithm2e}

\usepackage{amsmath, amsthm, amssymb, amsfonts}
\usepackage{subfig}

\usepackage{natbib}

%------------------------------------
%----------Configurations------------
%------------------------------------

\usetheme{Madrid}
\usecolortheme{default}
\useinnertheme{circles}

\definecolor{FirstColor}{rgb}{0.0157,0.2392,0.4902}
\definecolor{SecondColor}{rgb}{0.0157, 0.549, 0.8}

\setbeamertemplate{itemize items}[triangle]

\setbeamercolor*{palette primary}{bg=FirstColor, fg=white}
\setbeamercolor*{palette secondary}{bg=SecondColor, fg=white}
\setbeamercolor*{palette tertiary}{bg=white, fg=FirstColor}
\setbeamercolor*{palette quaternary}{bg=FirstColor,fg=white}
\setbeamercolor{structure}{fg=FirstColor}
\setbeamercolor{section in toc}{fg=FirstColor}

\hypersetup{colorlinks=true,citecolor=blue, urlcolor = cyan, linkcolor=blue}

\apptocmd{\frame}{}{\justifying}{}

%---------------------------------------------------
%------------------Itemize--------------------------
%---------------------------------------------------

\makeatletter
\newcommand{\my@beamer@setsep}{%
\ifnum\@itemdepth=1\relax
     \setlength\itemsep{\my@beamer@itemsepi}% separation for first level
   \else
     \ifnum\@itemdepth=2\relax
       \setlength\itemsep{\my@beamer@itemsepii}% separation for second level
     \else
       \ifnum\@itemdepth=3\relax
         \setlength\itemsep{\my@beamer@itemsepiii}% separation for third level
   \fi\fi\fi}
\newlength{\my@beamer@itemsepi}\setlength{\my@beamer@itemsepi}{3ex}
\newlength{\my@beamer@itemsepii}\setlength{\my@beamer@itemsepii}{1.5ex}
\newlength{\my@beamer@itemsepiii}\setlength{\my@beamer@itemsepiii}{1.5ex}
\newcommand\setlistsep[3]{%
    \setlength{\my@beamer@itemsepi}{#1}%
    \setlength{\my@beamer@itemsepii}{#2}%
    \setlength{\my@beamer@itemsepiii}{#3}%
}
\xpatchcmd{\itemize}
  {\def\makelabel}
  {\my@beamer@setsep\def\makelabel}
 {}
 {}

\xpatchcmd{\beamer@enum@}
  {\def\makelabel}
  {\my@beamer@setsep\def\makelabel}
 {}
 {}
\makeatother

%---------------------------------------------------
%-----------------Definitions-----------------------
%---------------------------------------------------

\newcommand{\Space}{\vspace{3ex}}
\newcommand{\R}{\mathbb{R}}
\newcommand{\onevec}{\boldsymbol{e}}
\newtheorem{proposition}[theorem]{Proposição}

%---------------------------------------------------
%----------------Front page-------------------------
%---------------------------------------------------

\title[Network Science]
{Rede de Personagens da Turma da Mônica}
%\subtitle{}
\pdfstringdefDisableCommands{%
  \def\\{}%
  \def\texttt#1{<#1>}%
}
\author[João Primaki e Igor Michels]
{
    Igor Patrício Michels \\
    João Vinícius Primaki Prado \\
    Prof: Alberto Paccanaro
}
\institute[]
{
  Escola de Matemática Aplicada \\
  Fundação Getulio Vargas
}
\date[\today]
{\today}

\titlegraphic{
    \vspace*{0.3cm}
    \hspace*{8.7cm}
    \includegraphics[width=.3\textwidth]{img/logo-emap.png}
}

\begin{document}

\begin{frame}
\titlepage
\end{frame} % capa da apresentação

\begin{frame}{Turma da Mônica / Monica's Gang / La Banda di Monica}
\begin{figure}
    \centering
    \includegraphics[scale = 0.45]{img/tdm.jpg}
\end{figure}
\end{frame}

\begin{frame}{Rede}
\begin{itemize}
    \item Rede da Turma da Mônica (Clássica e Jovem):
    \begin{itemize}
        \vspace{12pt}
        \item Nós: Personagens;
        \vspace{12pt}
        \item Arestas: Relação de aparecer numa mesma página.
    \end{itemize}
    \vspace{24pt}
        
    \item No momento temos aproximadamente 600 nós e 3200 arestas.
    % \item Estimativa de 1000 nós e 5000 arestas.
    % atualmente (7 gibis da TDM e 3 da TMJ) tem 500 nós e 2658 arestas
    % acho que é um chute baixo ainda, principalmente por que estamos resetando
    % as crianças e figurantes
\end{itemize}
\end{frame}

\begin{frame}{Rede}
\begin{figure}
    \centering
    \includegraphics[scale = 0.4]{img/graph.png}
\end{figure}
\end{frame}

\begin{frame}{Dados}
\begin{itemize}
    \item Motivação;
    % Turma da Mônica é basicamente um patrimônio nacional que moldou várias gerações. Existe a mais de 60 anos, já vendeu ao todo mais de 1 bilhão de gibis e esta presente em mais de 40 países com 14 idiomas distintos.
    \vspace{24pt}
        
    \item Coleta;
    % a coleta dos dados se dá manualmente, por meio da leitura e anotações dos personagens de cada página
    \vspace{24pt}
        
    \item Leitura e codificação dos dados.
    % estamos trabalhando em python, usando pandas para armazenar os dados dos gibis e depois passando os mesmos para csv, de modo a ter "uma database". Por fim utilizamos o networkx para trabalhar com a rede
\end{itemize}
\end{frame}

\begin{frame}{Gibi}
\begin{figure}
    \centering
    \includegraphics[scale = 0.25]{img/page.pdf}
    \includegraphics[scale = 0.25]{img/page.pdf}
\end{figure}
\end{frame}

\begin{frame}{Perguntas}
\begin{itemize}
    \item Qual a distância média entre as personagens?
    \vspace{12pt}
    
    \item Com exceção das personagens principais, quem seria o maior HUB da rede?
    \vspace{12pt}
    
    \item Qual a maior distância entre as personagens?
    % Entre quais personagens existe essa distância?
    % diâmetro da rede
    \vspace{12pt}
    
    % \item Quais as personagens que são mais próximas umas das outras?
    \item Qual o melhor amigo de cada personagem?
    % i.e. aresta de maior peso que sai desse personagem
    \vspace{12pt}
    
    \item Algoritmos de classificação de comunidades funcionam bem pra separar a rede nas turmas pré-definidas?
\end{itemize}
\end{frame}

\begin{frame}{Possíveis Pesquisas}
\begin{itemize}
    \item Correlação entre vendas e interação entre personagens
    % Descobrir quantas cópias foram vendidas de cada edição e verificar se existe alguma correlação entre as relações de personagens no gibi e o número de vendas. ou popularidade no scoobi/guia dos quadrinhos/gibiteca
\end{itemize}
\end{frame}

\begin{frame}{Agradecimentos}
\centering
Obrigado pela atenção!
\end{frame}

\end{document}